%=============================Preamble=============================%
\documentclass[10pt,a4paper]{article}
\usepackage[T1]{fontenc}
\usepackage[utf8]{inputenc}
\usepackage[brazil]{babel}
%\usepackage[math]{anttor} % fonte um pouco mais estilizada
\everymath{\displaystyle}
\usepackage{import}
%\usepackage{parskip}
%=========================Packages==================================%
\usepackage{color,lscape, amsmath, hyperref, booktabs, latexsym, multicol, gensymb, lmodern, natbib, graphicx, tikz, tkz-euclide, amssymb, enumitem, fancyhdr, lipsum, siunitx, setspace}


% configurações das questões, bem como: pontuação e estrutura.

\usepackage{tasks} % cria lista curta
\usepackage{exsheets} % cria questoes
\SetupExSheets[points]{name=ponto/s,number-format=\color{blue}} % define as configurações de pontuação das questões, e a cor da pontuação.

\DeclareInstance{exsheets-heading}{fancy-wp}{default}{
toc-reversed = true ,
indent-first = true ,
vscale = 2 ,
pre-code = \rule{\linewidth}{1pt} ,
post-code = \rule{\linewidth}{1pt} ,
title-format = \large\scshape\color{rgb:red,0.65;green,0.04;blue,0.07} ,
number-format = \large\bfseries\color{rgb:red,0.02;green,0.04;blue,0.48} ,
points-format = \itshape ,
points-pre-code = ( ,
points-post-code = ) ,
join =
{
number[r,B]title[l,B](.333em,0pt) ;
number[r,B]points[l,B](.333em,0pt)
} ,
attach = { main[hc,vc]number[hc,vc](0pt,0pt) }
}

%\SetupExSheets{headings=fancy-wp} % estilo diferente para o topo do enunciado com o nome " Exercício
%===========================Margins==============================%
\usepackage[top=8mm, bottom=20mm, left=8mm, right=8mm]{geometry}

%======================Cabeçalho e Rodapé========================%
\pagestyle{fancy}
\lfoot{\notaesquerda}
\cfoot{\thepage}
\rfoot{\notadireita}
%\lhead{HELLO}
%\chead{HELLO}
%\rhead{\textbf{The performance of new graduates}}
%\renewcommand{\headrulewidth}{0.4pt} %linha horizontal no topo da pagina
\renewcommand{\footrulewidth}{0.4pt} %linha horizontal no pé da pagina

\setlength\parindent{0pt}
\setlength\parskip{1.5ex}
\setlength\parsep{1.5\parskip}
%\thispagestyle{empty}%\bigskip %Rodapé na primeira pagina


%=======================informações da atividade===============================%
\newcommand{\atv}{Lista de Exercícios 04 -- Equação Polinomial do 2º Grau}
\newcommand{\bolsistas}{Lucas Felinto  e Matheus Jonatha – \atv \\ \preceptor}
\newcommand{\preceptor}{Professor Supervisor: Aqui vai o nome do professor}

\begin{document}

\frenchspacing
\begin{center}
    \begin{minipage}{4.3cm}
		\begin{center}
			\includegraphics[width=4.7cm, height=2.5cm]{logo_pibid.png}
		\end{center}
	\end{minipage}
	\begin{minipage}{14.5cm}
		\begin{center}
				{  \small \textbf{Instituto Federal de Educação, Ciência e Tecnologia do Rio Grande do Norte – IFRN \\
Coordenação de Aperfeiçoamento de Pessoal de Nível Superior – CAPES\\
Programa Institucional de Bolsa de Iniciação à Docência – PIBID\\
Subprojeto Matemática / Campus Natal Central}
                        
                }
		\end{center}
	\end{minipage}
	
\end{center}


{\sf
  \begin{center}
     \textbf{\rule{20cm}{0.4pt} \\ \vspace{3mm} \bolsistas  \\ \rule{20cm}{0.4pt} 
     }
  \end{center}
}\bigskip
\vspace{2mm}
\setlength{\marginparwidth}{5cm}
\small \noindent \textbf{Nome:}\hspace{0.3cm}\hrulefill \hrulefill
\hrulefill \hspace{0.1cm} 
\textbf{Número:}\hspace{0.1cm}\rule{1cm}{.1mm}


	\begin{center}
		\textsc{obs: faça todos os cálculos e todas as etapas, até para as questões de múltipla escolha. \\ será considerado para correção todos os passos do desenvolvimento do cálculo. \\ use as propriedades abordadas em sala ou explique discursivamente.}    
	\end{center}

%\begin{center}
%\textsc{\Large Exercícios}    %Titulo do topo, antes de iniciar as questões
%\end{center}

	\begin{multicols}{2}
	\setlength\columnseprule{1pt} % linha vertical entre as colunas
	
	
	    \import{questions/}{q1}
	    \import{questions/}{q2}
	    \import{questions/}{q3}
	    \import{questions/}{q4}
	    \import{questions/}{q5}
	     %\newpage %% ou \clearpage ou %% \pagebreak %% força uma quebra de pagina. caso os exercicios ocupem apenas metade de uma pagina.
	    \import{questions/}{q6}
	    \import{questions/}{q7}
	    \import{questions/}{q8}
	    \import{questions/}{q9}
	    
		%\begin{question}[type=exam] (\addpoints{10}) %pontuação da questão %caso queira que o enunciado tenha o nome "Exercício 1." e não "Questão", apague o a configuração [type=exam], caso queira personalizar o nome do jeito que quiser, use [name = NomeQueEuQueroVaiAqui], exemplo [name = Problema]
Escreva a equação $ax^2+bx+c=0$, para:

\begin{tasks}(1)
        \task $a=3$ ; $b=-2$ e $c=1$  
        \task $a=-1$ ; $b=0$ e $c=7$  
        \task $a=1$ ; $b=-5$ e $c=-6$  
    \end{tasks}
\end{question}


		%\begin{question}[type=exam] (\addpoints{10}) %pontuação da questão %caso queira que o enunciado tenha o nome "Exercício 1." e não "Questão", apague o a configuração [type=exam].
A figura~\ref{fig:graf1} representa o gráfico de uma função polinomial do 2º grau, escreva a função de descreve seu comportamento:

    \begingroup %adicionar imagem na questão, graficos e afins.
        \centering
        \includegraphics[width=0.4\linewidth ,height=0.4\linewidth]{graf1.png}
        \captionof{figure}{Gráfico de função}\label{fig:graf1}
    \endgroup

%\begin{tasks}(1)
 %       \task $x^2-1=2$
  %      \task $7x-1=0$
   %     \task $2x^2-2=0$   
%    \end{tasks}
\end{question}


   
		%\begin{question}[type=exam] (\addpoints{30}) %pontuação da questão %caso queira que o enunciado tenha o nome "Exercício 1." e não "Questão", apague o a configuração [type=exam].
Classifique cada equação do 2$^\circ$ grau em completa ou incompleta.
    \begin{tasks}(1)
       \task $x^2-3=0$
        \task $-9x^2+2x+6=0$
        \task $2x^2-30=0$
    \end{tasks}
\end{question}
		%\begin{question} %(\addpoints{50}) %pontuação da questão
	
Sabendo que $2$ é raiz da equação $(2p-1)x^2 -2px-2=0$, qual o valor de $p$.

\end{question}
		%\begin{question}[type=exam] (\addpoints{50})%pontuação da questão %caso queira que o enunciado tenha o nome "Exercício 1." e não "Questão", apague o a configuração [type=exam].

Identifique os \textbf{coeficientes} e calcule o \textbf{discriminante} para cada equação.
\begin{tasks}(2)
        \task $2x^2-11x+5=0$
        \task $2x^2+4x+4=0$
        \task $4-5x^2+2x=0$
        \task $4x^2+2x+1=0$
        \task $x^2+8x+16=0$
    \end{tasks}
\end{question}
		%\begin{question} 
Considere a equação $x^2+12x-189=0$, faça o que se pede abaixo:
\begin{tasks}
        \task Identifique os coeficientes $a$, $b$ e $c$.
        \task Calcule o valor do Delta
        \task Determine o valor de $x_{1}$ e $x_{2}$ (Usando formula de Bhaskara)
    \end{tasks}
\end{question}
		%\begin{question} (\addpoints{80}) %pontuação da questão
Classifique as afirmações em \textbf{V}(verdadeiro) ou \textbf{F} (falso)

\begin{itemize}
    \item [(~~)] Se o discriminante da equação é igual a zero, ela tem duas raízes reais e iguais.
    \item [(~~)] Se o discriminante da equação é menor que zero, ela tem duas raízes reais diferentes.
    \item [(~~)] Se o discriminante da equação é maior que zero, ela tem duas raízes reais e diferentes.
    \item [(~~)] Se o discriminante da equação é igual a zero, ela não tem raízes reais.
\end{itemize}
\end{question}

		%\begin{question}[type=exam] (\addpoints{20}) %pontuação da questão %caso queira que o enunciado tenha o nome "Exercício 1." e não "Questão", apague o a configuração [type=exam], caso queira personalizar o nome do jeito que quiser, use [name = NomeQueEuQueroVaiAqui], exemplo [name = Problema]
Determine as raízes reais das equações incompletas:

\begin{tasks}(2)
        \task $x^2-5x=0$
        \task $-x^2+12x=0$
        \task $x^2-9=0$
        \task $25x^2-1=0$
        \task $(x-7) \cdot (x-3)+10x=30$
    \end{tasks}
\end{question}

		%\begin{question}[type=exam] (\addpoints{20}) %pontuação da questão %caso queira que o enunciado tenha o nome "Exercício 1." e não "Questão", apague o a configuração [type=exam].

 Resolva as equações completas no conjunto $\mathbb{R}$:
\begin{tasks}
        \task $4x^2-4x+1=0$
        \task $x^2+6x+9=0$
    \end{tasks}
\end{question}

	
	\end{multicols}
        
          \import{questions/}{q7} % exemplo para mostrar que pode colocar questões fora das colunas e mesclar os estilos. Recomendado adicionar questões que incluem imagens, ao final e fora das colunas.
          
          
          


\end{document}