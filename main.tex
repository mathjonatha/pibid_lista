%=============================Preamble=============================%
\documentclass[10pt,a4paper]{article}
\usepackage[T1]{fontenc}
\usepackage[utf8]{inputenc}
\usepackage[brazil]{babel}
%\usepackage[math]{anttor} % fonte um pouco mais estilizada
\everymath{\displaystyle}
\usepackage{import}
%\usepackage{parskip}
%=========================Packages==================================%
\usepackage{color,lscape, amsmath, hyperref, booktabs, latexsym, multicol, gensymb, lmodern, natbib, graphicx, tikz, tkz-euclide, amssymb, enumitem, fancyhdr, lipsum, siunitx, setspace}


% configurações das questões, bem como: pontuação e estrutura.

\usepackage{tasks} % cria lista curta
\usepackage{exsheets} % cria questoes
\SetupExSheets[points]{name=ponto/s,number-format=\color{blue}} % define as configurações de pontuação das questões, e a cor da pontuação.

\DeclareInstance{exsheets-heading}{fancy-wp}{default}{
toc-reversed = true ,
indent-first = true ,
vscale = 2 ,
pre-code = \rule{\linewidth}{1pt} ,
post-code = \rule{\linewidth}{1pt} ,
title-format = \large\scshape\color{rgb:red,0.65;green,0.04;blue,0.07} ,
number-format = \large\bfseries\color{rgb:red,0.02;green,0.04;blue,0.48} ,
points-format = \itshape ,
points-pre-code = ( ,
points-post-code = ) ,
join =
{
number[r,B]title[l,B](.333em,0pt) ;
number[r,B]points[l,B](.333em,0pt)
} ,
attach = { main[hc,vc]number[hc,vc](0pt,0pt) }
}

%\SetupExSheets{headings=fancy-wp} % estilo diferente para o topo do enunciado com o nome " Exercício
%===========================Margins==============================%
\usepackage[top=8mm, bottom=20mm, left=8mm, right=8mm]{geometry}

%======================Cabeçalho e Rodapé========================%
\pagestyle{fancy}
\lfoot{\notaesquerda}
\cfoot{\thepage}
\rfoot{\notadireita}
%\lhead{HELLO}
%\chead{HELLO}
%\rhead{\textbf{The performance of new graduates}}
%\renewcommand{\headrulewidth}{0.4pt} %linha horizontal no topo da pagina
\renewcommand{\footrulewidth}{0.4pt} %linha horizontal no pé da pagina

\setlength\parindent{0pt}
\setlength\parskip{1.5ex}
\setlength\parsep{1.5\parskip}
%\thispagestyle{empty}%\bigskip %Rodapé na primeira pagina


\graphicspath{{imgs/}} %informa a pasta em que as imagens estão
\usepackage{capt-of}%%To get the caption
%=======================informações da atividade===============================%
\newcommand{\atv}{Lista de Exercícios 04 -- Equação Polinomial do 2º Grau}
\newcommand{\preceptor}{Professor Supervisor: Aqui vai o nome do professor}
\newcommand{\turma}{1º série - noturno}
\newcommand{\bolsistas}{Nome do bolsista 1  e Nome do bolsista 2 – \atv \\ \preceptor}
\hypersetup{pdftitle={Modelo de lista - PIBID},pdfauthor={Matheus Jonatha}}
%=====================informações de rodapé=================%
\newcommand{\notaesquerda}{PIBID Matemática CNAT\\ \totalpoints % mostra a soma das pontuações distribuídas nas questões.
}
\newcommand{\notadireita}{Nome da Instituição em que atua \\ \turma}


\begin{document}

\frenchspacing
\begin{center}
    \begin{minipage}{4.3cm}
		\begin{center}
			\includegraphics[width=4.7cm, height=2.5cm]{logo_pibid.png}
		\end{center}
	\end{minipage}
	\begin{minipage}{14.5cm}
		\begin{center}
				{  \small \textbf{Instituto Federal de Educação, Ciência e Tecnologia do Rio Grande do Norte – IFRN \\
Coordenação de Aperfeiçoamento de Pessoal de Nível Superior – CAPES\\
Programa Institucional de Bolsa de Iniciação à Docência – PIBID\\
Subprojeto Matemática / Campus Natal Central}
                        
                }
		\end{center}
	\end{minipage}
	
\end{center}


{\sf
  \begin{center}
     \textbf{\rule{20cm}{0.4pt} \\ \vspace{3mm} \bolsistas  \\ \rule{20cm}{0.4pt} 
     }
  \end{center}
}\bigskip
\vspace{2mm}
\setlength{\marginparwidth}{5cm}
\small \noindent \textbf{Nome:}\hspace{0.3cm}\hrulefill \hrulefill
\hrulefill \hspace{0.1cm} 
\textbf{Número:}\hspace{0.1cm}\rule{1cm}{.1mm}


	\begin{center}
		\textsc{obs: faça todos os cálculos e todas as etapas, até para as questões de múltipla escolha. \\ será considerado para correção todos os passos do desenvolvimento do cálculo. \\ use as propriedades abordadas em sala ou explique discursivamente.}    
	\end{center}

%\begin{center}
%\textsc{\Large Exercícios}    %Titulo do topo, antes de iniciar as questões
%\end{center}

	\begin{multicols}{2}
	\setlength\columnseprule{0.6pt} % linha vertical entre as colunas
	
	    \import{questions/}{q1}
	    \import{questions/}{q2}
	    \import{questions/}{q3}
	    \import{questions/}{q4}
	    \import{questions/}{q5}
	     %\newpage %% ou \clearpage ou %% \pagebreak %% força uma quebra de pagina. caso os exercicios ocupem apenas metade de uma pagina.
	    \import{questions/}{q6}
	    \import{questions/}{q7}
	    \import{questions/}{q8}
	    \import{questions/}{q9}
	    
	
	\end{multicols}
        
        %  \import{questions/}{q7} % exemplo para mostrar que pode colocar questões fora das colunas e mesclar os estilos. Recomendado adicionar questões que incluem imagens, ao final e fora das colunas.
          
          
          


\end{document}