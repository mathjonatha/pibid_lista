\usepackage[T1]{fontenc}
\usepackage[utf8]{inputenc}
\usepackage[brazil]{babel}
%\usepackage[math]{anttor} % fonte um pouco mais estilizada
\usepackage{tasks} % cria lista curta
\usepackage{exsheets} % cria questoes
\everymath{\displaystyle} 
%\usepackage{parskip}
\graphicspath{{imgs/}} %informa a pasta em que as imagens estão

%=========================Packages==================================%
\usepackage{color,lscape, amsmath, hyperref, booktabs, latexsym, multicol, gensymb, lmodern, natbib, graphicx, tikz, tkz-euclide, amssymb, enumitem, fancyhdr, lipsum, siunitx, setspace}

%===========================Margins==============================%
\usepackage[top=8mm, bottom=20mm, left=8mm, right=8mm]{geometry}

%======================Cabeçalho e Rodapé========================%
\pagestyle{fancy}
\lfoot{PIBID Matemática CNAT}
\cfoot{\thepage}
\rfoot{Instituto Padre Miguelinho}
%\lhead{HELLO}
%\chead{HELLO}
%\rhead{\textbf{The performance of new graduates}}
%\renewcommand{\headrulewidth}{0.4pt} %linha horizontal no topo da pagina
\renewcommand{\footrulewidth}{0.4pt} %linha horizontal no pé da pagina

\setlength\parindent{0pt}
\setlength\parskip{1.5ex}
\setlength\parsep{1.5\parskip}
%\thispagestyle{empty}%\bigskip %Rodapé na primeira pagina
