\usepackage[T1]{fontenc}
\usepackage[utf8]{inputenc}
\usepackage[brazil]{babel}
%\usepackage[math]{anttor} % fonte um pouco mais estilizada
\everymath{\displaystyle}
\usepackage{import}
%\usepackage{parskip}
%=========================Packages==================================%
\usepackage{color,lscape, amsmath, hyperref, booktabs, latexsym, multicol, gensymb, lmodern, natbib, graphicx, tikz, tkz-euclide, amssymb, enumitem, fancyhdr, lipsum, siunitx, setspace}


% configurações das questões, bem como: pontuação e estrutura.

\usepackage{tasks} % cria lista curta
\usepackage{exsheets} % cria questoes
\SetupExSheets[points]{name=ponto/s,number-format=\color{blue}} % define as configurações de pontuação das questões, e a cor da pontuação.

\DeclareInstance{exsheets-heading}{fancy-wp}{default}{
toc-reversed = true ,
indent-first = true ,
vscale = 2 ,
pre-code = \rule{\linewidth}{1pt} ,
post-code = \rule{\linewidth}{1pt} ,
title-format = \large\scshape\color{rgb:red,0.65;green,0.04;blue,0.07} ,
number-format = \large\bfseries\color{rgb:red,0.02;green,0.04;blue,0.48} ,
points-format = \itshape ,
points-pre-code = ( ,
points-post-code = ) ,
join =
{
number[r,B]title[l,B](.333em,0pt) ;
number[r,B]points[l,B](.333em,0pt)
} ,
attach = { main[hc,vc]number[hc,vc](0pt,0pt) }
}

%\SetupExSheets{headings=fancy-wp} % estilo diferente para o topo do enunciado com o nome " Exercício
%===========================Margins==============================%
\usepackage[top=8mm, bottom=20mm, left=8mm, right=8mm]{geometry}

%======================Cabeçalho e Rodapé========================%
\pagestyle{fancy}
\lfoot{\notaesquerda}
\cfoot{\thepage}
\rfoot{\notadireita}
%\lhead{HELLO}
%\chead{HELLO}
%\rhead{\textbf{The performance of new graduates}}
%\renewcommand{\headrulewidth}{0.4pt} %linha horizontal no topo da pagina
\renewcommand{\footrulewidth}{0.4pt} %linha horizontal no pé da pagina

\setlength\parindent{0pt}
\setlength\parskip{1.5ex}
\setlength\parsep{1.5\parskip}
%\thispagestyle{empty}%\bigskip %Rodapé na primeira pagina

