\begin{question}[type=exam] (\addpoints{10}) %pontuação da questão %caso queira que o enunciado tenha o nome "Exercício 1." e não "Questão", apague o a configuração [type=exam], caso queira personalizar o nome do jeito que quiser, use [name = NomeQueEuQueroVaiAqui], exemplo [name = Problema]
Classifique as afirmações em \textbf{V}(verdadeiro) ou \textbf{F} (falso)

\begin{itemize}
    \item [(~~)] Se o discriminante da equação é igual a zero, ela tem duas raízes reais e iguais.
    \item [(~~)] Se o discriminante da equação é menor que zero, ela tem duas raízes reais diferentes.
    \item [(~~)] Se o discriminante da equação é maior que zero, ela tem duas raízes reais e diferentes.
    \item [(~~)] Se o discriminante da equação é igual a zero, ela não tem raízes reais.
\end{itemize}
\end{question}
